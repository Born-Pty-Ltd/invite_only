\section{Requirements} \label{Requirements}

\subsection{Functional Requirements} \label{Functional Requirements}
\begin{itemize}
    \item[R1.] As a manager I want to control access to one or more areas so that only people allowed to enter are given access.
    \begin{itemize}
        \item[R1.1.] As a manager I want to be able to create a profile on "Invite Only" so that I can access my data from anywhere.
        \item[R1.2.] As a manager I want to to create, edit and delete access-controlled spaces so that I can associate digital spaces with physical spaces.
        \begin{itemize}
            \item[R1.2.1.] As a manager I want to specify a name for an access-controlled space so that the space is identifiable.
            \item[R1.2.2.] As a manager I want to set a minimum age for an access-controlled space so that I can prevent people below this age from entering the space.
            \item[R1.2.3.] As a manager I want to set the maximum number of people allowed inside an access-controlled space so that the  number of people inside a space is limited.
            \item[R1.2.4.] As a manager I want to mark a space as "Invite Only" so that only people with invitations may enter a space.
            \item[R1.2.5.] As a manager, I want to define a list of Inviters for an "Invite Only" space so that these users can also send out invites.
            \item[R1.2.6.] As a manager I want to add Guards to an space so that they may assist me in scanning identification documents.
        \end{itemize}
        \item[R1.3.] As a manager I want to view metrics associated with my access-controlled spaces.
        \begin{itemize}
            \item[1.3.1.] As a manager I want to see a log of who entered and who left an access-controlled space.
            \item[1.3.2.] As a manager I want a view of who was inside an access-controlled space during a defined period.
        \end{itemize}
    \end{itemize}
    
    \newpage
    
    \item[R2.] As a guard I want to scan identity documents to decide whether people are allowed to enter access-controlled spaces.
    \begin{itemize}
        \item[R2.1.] As a guard I want to create a profile on "Invite Only" so that I can be added as a guard to a space defined by my manager.
        \item[R2.2.] As a guard I want to scan identity documents to find out if the owners of the documents need to be given access to a space.
    \end{itemize}
    
    \item[R3.] As an Inviter, I want to send my guests an invite allowing them to enter an access-controlled space.
    \begin{itemize}
        \item[R3.1.] As an Inviter I want to create a profile on "Invite Only" so that I can be added as an Inviter to a space defined by someone else.
        \item[R3.2.] As an Inviter I want to generate and share invite codes with my guests so that they may use their identity documents to enter an access-controlled space.
    \end{itemize}
    
    \item[R4.] As an invitee I want to be able to enter the area with my identification document.
    \begin{itemize}
        \item[R4.1.] As an invitee, if I am not already register on "Invite Only", I want to register on the app so that I can link my identity documents.
        \item[R4.2.] As an invitee I want to scan and link my identity documents so that I can use them for entering access-controlled spaces.
        \item[R4.3.] As an invitee I want to submit an invite code that has been given to me so that I may gain access to an access-controlled space.
    \end{itemize}
    
    \item[R5.] As any user of "Invite Only", I want to be able to delete my profile so that I retain the right to have my data removed as defined in the Protection of Personal Information Act.
    
\end{itemize}

\newpage

\subsection{Non-Functional Requirements} \label{Non-Functional Requirements}

This section details quantifiable quality attributes for the Invite Only system. The architectural approaches taken to address these requirements are detailed in section \ref{Architecture}.

\subsubsection{Performance and Scalability}
When using the "Invite Only" system over an LTE connection, users should never need to wait longer than 3 seconds for a response to an action. Additionally, changes by any user need to be reflected on a real-time basis to all other users to whom the changes are relevant. Finally, the aforementioned requirements need to hold regardless of the amount of data stored on the system or the number of users actively using the system at any given time.

\subsubsection{Portability and Compatibility}
The "Invite Only" mobile application needs to be accessible by at least 90\% of users of smartphones produced by any manufacturer. Consequently, the application must be compatible with all versions of Android since version 5.0 and all versions of iOS since version 12.0.

\subsubsection{Reliability, Availability and Maintainability}
The "Invite Only" system shall not be offline for more than 1 minute per year of operation - this means the system must be available to all users 99.999\% of the time. Furthermore, maintenance required to fix defects must occur within 48 hours of detecting defects and be allowed to take place without bringing the system down completely.

\subsubsection{Security}
Passwords entered on the system shall never be visible at the point of entry or at any other time. User's must also be required to use biometric authentication when an action effects the access of an access-controlled space. All data access which exposes or alters sensitive user or access-controlled space information must be protected using the OAuth 2.0. protocol.

\newpage

\subsubsection{Usability}
All main actions must also be accessible within 2 taps or clicks after opening the "Invite Only" application and user's must be able to learn how to complete these main actions within 15 minutes of completing registration. User's must also have the ability to overcome any error within a single step after being presented with the error.

\subsubsection{Localization}
As the application will be dealing with identification documents in a South African context, compliance with the Protection of Personal Information (PoPI) Act is essential. Users must be allowed to control the access, storage and integrity of their personal information at all times.

\newpage

\subsection{Technology Requirements} \label{Technology Requirements}

\subsubsection{Flutter}

Flutter, Google’s UI toolkit for building natively compiled applications for mobile, web, and desktop from a single codebase, will be used for the implementation of the Invite Only mobile application. 

The choice of Flutter is motivated by the need to build a multi-platform application from a single codebase and the native performance benefits that Flutter has over other multi-platform frameworks, like React Native. These considerations relate predominantly to the performance, portability, availability and maintainability requirements listed in section \ref{Non-Functional Requirements}.

\subsubsection{Cloud Firestore}

To store data across Invite Only applications running on different devices, Cloud Firestore will be used. Cloud Firestore is a flexible, scalable NoSQL database for mobile, web, and server development.

Firstly - through automatic multi-region data replication, strong consistency guarantees, atomic batch operations, and real transaction support - using Cloud Firestore addresses the scalability, reliability, reliability and availability requirements detailed in section \ref{Non-Functional Requirements}.

Additionally, data stored in the Cloud Firestore database is protected using industry standards like OAuth 2.0. This prevents  any unauthorized access to the data and corresponds with the Security requirements included in section \ref{Non-Functional Requirements}.

\subsubsection{Firebase Authentication}

In line with both the functional and non-functional requirements detailed in sections \ref{Functional Requirements} and \ref{Non-Functional Requirements}, managing user identities is an integral part of the Invite Only system.

As such, Firebase Authentication will be used for the authentication of users. Firebase Authentication supports authentication using passwords, phone numbers, popular federated identity providers like Google, Facebook and Twitter, and more. Also, it integrates tightly with the other Firebase Services mentioned as technical requirements.